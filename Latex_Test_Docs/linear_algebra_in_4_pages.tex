Linear algebra explained in four pages
Excerpt from the N O BULLSHIT GUIDE TO LINEAR ALGEBRA by Ivan Savov
Abstract —This document will review the fundamental ideas of linear algebra.
We will learn about matrices, matrix operations, linear transformations and
discuss both the theoretical and computational aspects of linear algebra. The
tools of linear algebra open the gateway to the study of more advanced
mathematics. A lot of knowledge buzz awaits you if you choose to follow the
path of understanding , instead of trying to memorize a bunch of formulas.
I. I NTRODUCTION
Linear algebra is the math of vectors and matrices. Let nbe a positive
integer and let Rdenote the set of real numbers, then Rnis the set of all
n-tuples of real numbers. A vector ~ v2Rnis ann-tuple of real numbers.
The notation “2S” is read “element of S.” For example, consider a vector
that has three components:
~ v= \left(v1;v2;v3\right)2\left(R;R;R\right)R3:
A matrixA2Rmnis a rectangular array of real numbers with mrows
andncolumns. For example, a 32matrix looks like this:
A=2
4a11a12
a21a22
a31a323
522
4R R
R R
R R3
5R32:
The purpose of this document is to introduce you to the mathematical
operations that we can perform on vectors and matrices and to give you a
feel of the power of linear algebra. Many problems in science, business,
and technology can be described in terms of vectors and matrices so it is
important that you understand how to work with these.
Prerequisites
The only prerequisite for this tutorial is a basic understanding of high school
math concepts1like numbers, variables, equations, and the fundamental
arithmetic operations on real numbers: addition \left(denoted \+\right), subtraction
\left(denoted\right), multiplication \left(denoted implicitly\right), and division \left(fractions\right).
You should also be familiar with functions that take real numbers as
inputs and give real numbers as outputs, f:R!R. Recall that, by
definition, the inverse function f1undoes the effect of f. If you are
givenf\left(x\right)and you want to find x, you can use the inverse function as
follows:f1\left(f\left(x\right)\right) =x. For example, the function f\left(x\right) = ln\left(x\right)has the
inversef1\left(x\right) =ex, and the inverse of g\left(x\right) =pxisg1\left(x\right) =x2.
II. D EFINITIONS
A. Vector operations
We now define the math operations for vectors. The operations we can
perform on vectors ~ u= \left(u1;u2;u3\right)and~ v= \left(v1;v2;v3\right)are: addition,
subtraction, scaling, norm \left(length\right), dot product, and cross product:
~ u\+~ v= \left(u1\+v1;u2\+v2;u3\+v3\right)
~ u~ v= \left(u1v1;u2v2;u3v3\right)
~ u= \left(u1;u2;u3\right)
jj~ ujj=q
u2
1\+u2
2\+u2
3
~ u~ v=u1v1\+u2v2\+u3v3
~ u~ v=\left(u2v3u3v2; u3v1u1v3; u1v2u2v1\right)
The dot product and the cross product of two vectors can also be described
in terms of the angle between the two vectors. The formula for the dot
product of the vectors is ~ u~ v=k~ ukk~ vkcos. We say two vectors ~ uand
~ vareorthogonal if the angle between them is 90. The dot product of
orthogonal vectors is zero: ~ u~ v=k~ ukk~ vkcos\left(90\right) = 0 .
Thenorm of the cross product is given by k~ u~ vk=k~ ukk~ vksin. The
cross product is not commutative: ~ u~ v6=~ v~ u, in fact~ u~ v=~ v~ u.
1A good textbook to \left(re\right)learn high school math is minireference.comB. Matrix operations
We denote by Athe matrix as a whole and refer to its entries as aij.
The mathematical operations defined for matrices are the following:
addition \left(denoted \+\right)
C=A\+B,cij=aij\+bij:
subtraction \left(the inverse of addition\right)
matrix product. The product of matrices A2RmnandB2Rn`
is another matrix C2Rm`given by the formula
C=AB,cij=nX
k=1aikbkj;
2
4a11a12
a21a22
a31a323
5b11b12
b21b22
=2
4a11b11\+a12b21a11b12\+a12b22
a21b11\+a22b21a21b12\+a22b22
a31b11\+a32b21a31b12\+a32b223
5
matrix inverse \left(denoted A1\right)
matrix transpose \left(denotedT\right):
123
123T
=2
411
22
333
5:
matrix trace: Tr \left[A\right]Pn
i=1aii
\determinant \left(denoted \det \left(A\right)orjAj\right)
Note that the matrix product is not a commutative operation: AB6=BA.
C. Matrix-vector product
The matrix-vector product is an important special case of the matrix-
matrix product. The product of a 32matrixAand the 21column
vector~ xresults in a 31vector~ ygiven by:
~ y=A~ x,2
4y1
y2
y33
5=2
4a11a12
a21a22
a31a323
5x1
x2
=2
4a11x1\+a12x2
a21x1\+a22x2
a31x1\+a32x23
5
=x12
4a11
a21
a313
5\+x22
4a12
a22
a323
5 \left(C\right)
=2
4\left(a11;a12\right)~ x
\left(a21;a22\right)~ x
\left(a31;a32\right)~ x3
5: \left(R\right)
There are two2fundamentally different yet equivalent ways to interpret the
matrix-vector product. In the column picture, \left(C\right), the multiplication of the
matrixAby the vector ~ xproduces a linear combination of the columns
of the matrix :~ y=A~ x=x1A\left[:;1\right]\+x2A\left[:;2\right], whereA\left[:;1\right]andA\left[:;2\right]are
the first and second columns of the matrix A.
In the row picture, \left(R\right), multiplication of the matrix Aby the vector ~ x
produces a column vector with coefficients equal to the dot products of
rows of the matrix with the vector ~ x.
D. Linear transformations
The matrix-vector product is used to define the notion of a linear
transformation , which is one of the key notions in the study of linear
algebra. Multiplication by a matrix A2Rmncan be thought of as
computing a linear transformation TAthat takesn-vectors as inputs and
producesm-vectors as outputs:
TA:Rn!Rm:
2For more info see the video of Prof. Strang’s MIT lecture: bit.ly/10vmKcL
12
Instead of writing ~ y=TA\left(~ x\right)for the linear transformation TAapplied to
the vector~ x, we simply write ~ y=A~ x. Applying the linear transformation
TAto the vector ~ xcorresponds to the product of the matrix Aand the
column vector ~ x. We sayTAisrepresented by the matrixA.
You can think of linear transformations as “vector functions” and describe
their properties in analogy with the regular functions you are familiar with:
functionf:R!R,linear transformation TA:Rn!Rm
inputx2R,input~ x2Rn
outputf\left(x\right),outputTA\left(~ x\right) =A~ x2Rm
gf=g\left(f\left(x\right)\right),TB\left(TA\left(~ x\right)\right) =BA~ x
function inverse f1,matrix inverse A1
zeros off, N \left(A\right)null space of A
range off,C\left(A\right)column space of A=range ofTA
Note that the combined effect of applying the transformation TAfollowed
byTBon the input vector ~ xis equivalent to the matrix product BA~ x .
E. Fundamental vector spaces
Avector space consists of a set of vectors and all linear combinations of
these vectors. For example the vector space S=spanf~ v1;~ v2gconsists of
all vectors of the form ~ v=~ v1\+~ v2, whereandare real numbers.
We now define three fundamental vector spaces associated with a matrix A.
Thecolumn space of a matrixAis the set of vectors that can be produced
as linear combinations of the columns of the matrix A:
C\left(A\right)f~ y2Rmj~ y=A~ xfor some~ x2Rng:
The column space is the range of the linear transformation TA\left(the set
of possible outputs\right). You can convince yourself of this fact by reviewing
the definition of the matrix-vector product in the column picture \left(C\right). The
vectorA~ xcontainsx1times the 1stcolumn ofA,x2times the 2ndcolumn
ofA, etc. Varying over all possible inputs ~ x, we obtain all possible linear
combinations of the columns of A, hence the name “column space.”
Thenull spaceN\left(A\right)of a matrixA2Rmnconsists of all the vectors
that the matrix Asends to the zero vector:
N\left(A\right)
~ x2RnjA~ x=~0	
:
The vectors in the null space are orthogonal to all the rows of the matrix.
We can see this from the row picture \left(R\right): the output vectors is ~0if and
only if the input vector ~ xis orthogonal to all the rows of A.
The row space of a matrix A, denotedR\left(A\right), is the set of linear
combinations of the rows of A. The row spaceR\left(A\right)is the orthogonal
complement of the null space N\left(A\right). This means that for all vectors
~ v2R\left(A\right)and all vectors ~ w2N\left(A\right), we have~ v~ w= 0. Together, the
null space and the row space form the domain of the transformation TA,
Rn=N\left(A\right)R\left(A\right), wherestands for orthogonal direct sum .
F . Matrix inverse
By definition, the inverse matrix A1undoes the effects of the matrix A.
The cumulative effect of applying A1afterAis the identity matrix 1:
A1A= 12
41 0
...
0 13
5:
The identity matrix \left(ones on the diagonal and zeros everywhere else\right)
corresponds to the identity transformation: T 1\left(~ x\right) = 1~ x=~ x, for all~ x.
The matrix inverse is useful for solving matrix equations. Whenever we
want to get rid of the matrix Ain some matrix equation, we can “hit” A
with its inverse A1to make it disappear. For example, to solve for the
matrixXin the equation XA=B, multiply both sides of the equation
byA1from the right: X=BA1. To solve for XinABCXD =E,
multiply both sides of the equation by D1on the right and by A1,B1
andC1\left(in that order\right) from the left: X=C1B1A1ED1.III. C OMPUTATIONAL LINEAR ALGEBRA
Okay, I hear what you are saying “Dude, enough with the theory talk, let’s
see some calculations.” In this section we’ll look at one of the fundamental
algorithms of linear algebra called Gauss–Jordan elimination.
A. Solving systems of equations
Suppose we’re asked to solve the following system of equations:
1x1\+ 2x2= 5;
3x1\+ 9x2= 21:\left(1\right)
Without a knowledge of linear algebra, we could use substitution, elimina-
tion, or subtraction to find the values of the two unknowns x1andx2.
Gauss–Jordan elimination is a systematic procedure for solving systems
of equations based the following row operations :
\right) Adding a multiple of one row to another row
\right) Swapping two rows

\right) Multiplying a row by a constant
These row operations allow us to simplify the system of equations without
changing their solution.
To illustrate the Gauss–Jordan elimination procedure, we’ll now show the
sequence of row operations required to solve the system of linear equations
described above. We start by constructing an augmented matrix as follows:
12 5
3 9 21
:
The first column in the augmented matrix corresponds to the coefficients of
the variable x1, the second column corresponds to the coefficients of x2,
and the third column contains the constants from the right-hand side.
The Gauss-Jordan elimination procedure consists of two phases. During
the first phase, we proceed left-to-right by choosing a row with a leading
one in the leftmost column \left(called a pivot \right) and systematically subtracting
that row from all rows below it to get zeros below in the entire column. In
the second phase, we start with the rightmost pivot and use it to eliminate
all the numbers above it in the same column. Let’s see this in action.
1\right) The first step is to use the pivot in the first column to eliminate the
variablex1in the second row. We do this by subtracting three times
the first row from the second row, denoted R2 R23R1,
125
0 3 6
:
2\right) Next, we create a pivot in the second row using R2 1
3R2:
1 2 5
012
:
3\right) We now start the backward phase and eliminate the second variable
from the first row. We do this by subtracting two times the second
row from the first row R1 R12R2:1 0 1
0 1 2
:
The matrix is now in reduced row echelon form \left(RREF\right), which is its
“simplest” form it could be in. The solutions are: x1= 1,x2= 2.
B. Systems of equations as matrix equations
We will now discuss another approach for solving the system of
equations. Using the definition of the matrix-vector product, we can express
this system of equations \left(1\right) as a matrix equation:
1 2
3 9x1
x2
=5
21
:
This matrix equation had the form A~ x=~b, whereAis a22matrix,~ x
is the vector of unknowns, and ~bis a vector of constants. We can solve for
~ xby multiplying both sides of the equation by the matrix inverse A1:
A1A~ x= 1~ x=x1
x2
=A1~b=32
3
11
35
21
=1
2
:
But how did we know what the inverse matrix A1is?3
IV. C OMPUTING THE INVERSE OF A MATRIX
In this section we’ll look at several different approaches for computing
the inverse of a matrix. The matrix inverse is unique so no matter which
method we use to find the inverse, we’ll always obtain the same answer.
A. Using row operations
One approach for computing the inverse is to use the Gauss–Jordan
elimination procedure. Start by creating an array containing the entries
of the matrix Aon the left side and the identity matrix on the right side:
1 2 1 0
3 9 0 1
:
Now we perform the Gauss-Jordan elimination procedure on this array.
1\right) The first row operation is to subtract three times the first row from
the second row: R2 R23R1. We obtain:1 2 1 0
0 33 1
:
2\right) The second row operation is divide the second row by 3:R2 1
3R2
1 2 1 0
0 111
3
:
3\right) The third row operation is R1 R12R21 0 32
3
0 111
3
:
The array is now in reduced row echelon form \left(RREF\right). The inverse matrix
appears on the right side of the array.
Observe that the sequence of row operations we used to solve the specific
system of equations in A~ x=~bin the previous section are the same as the
row operations we used in this section to find the inverse matrix. Indeed,
in both cases the combined effect of the three row operations is to “undo”
the effects of A. The right side of the 24array is simply a convenient
way to record this sequence of operations and thus obtain A1.
B. Using elementary matrices
Every row operation we perform on a matrix is equivalent to a left-
multiplication by an elementary matrix . There are three types of elementary
matrices in correspondence with the three types of row operations:
R:R1 R1\+mR2,E=1m
0 1
R:R1$R2,E=0 1
1 0
R
:R1 mR1,E
=m0
0 1
Let’s revisit the row operations we used to find A1in the above section
representing each row operation as an elementary matrix multiplication.
1\right) The first row operation R2 R23R1corresponds to a multipli-
cation by the elementary matrix E1:
E1A=1 0
3 11 2
3 9
=1 2
0 3
:
2\right) The second row operation R2 1
3R2corresponds to a matrix E2:
E2\left(E1A\right) =1 0
01
31 2
0 3
=1 2
0 1
:
3\right) The final step, R1 R12R2, corresponds to the matrix E3:
E3\left(E2E1A\right) =12
0 11 2
0 1
=1 0
0 1
:
Note thatE3E2E1A= 1, so the product E3E2E1must be equal to A1:
A1=E3E2E1=12
0 11 0
01
31 0
3 1
=32
3
11
3
:
The elementary matrix approach teaches us that every invertible matrix
can be decomposed as the product of elementary matrices. Since we know
A1=E3E2E1thenA= \left(A1\right)1= \left(E3E2E1\right)1=E1
1E1
2E1
3.C. Using a computer
The last \left(and most practical\right) approach for finding the inverse of a matrix
is to use a computer algebra system like the one at live.sympy.org .
>>> A = Matrix\left( \left[\left[1,2\right],\left[3,9\right]\right] \right) # define A
\left[1, 2\right]
\left[3, 9\right]
>>> A.inv\left(\right) # calls the inv method on A
\left[ 3, -2/3\right]
\left[-1, 1/3\right]
You can use sympy to “check” your answers on homework problems.
V. O THER TOPICS
We’ll now discuss a number of other important topics of linear algebra.
A. Basis
Intuitively, a basis is any set of vectors that can be used as a coordinate
system for a vector space. You are certainly familiar with the standard basis
for thexy-plane that is made up of two orthogonal axes: the x-axis and
they-axis. A vector ~ vcan be described as a coordinate pair \left(vx;vy\right)with
respect to these axes, or equivalently as ~ v=vx^{\+vy^|, where ^{\left(1;0\right)
and^|\left(0;1\right)are unit vectors that point along the x-axis andy-axis
respectively. However, other coordinate systems are also possible.
Definition \left(Basis\right) .A basis for a n-dimensional vector space Sis any set
ofnlinearly independent vectors that are part of S.
Any set of two linearly independent vectors f^e1;^e2gcan serve as a basis
forR2. We can write any vector ~ v2R2as a linear combination of these
basis vectors ~ v=v1^e1\+v2^e2.
Note the same vector~ vcorresponds to different coordinate pairs depend-
ing on the basis used: ~ v= \left(vx;vy\right)in the standard basis Bsf^{;^|g, and
~ v= \left(v1;v2\right)in the basisBef^e1;^e2g. Therefore, it is important to keep
in mind the basis with respect to which the coefficients are taken, and if
necessary specify the basis as a subscript, e.g., \left(vx;vy\right)Bsor\left(v1;v2\right)Be.
Converting a coordinate vector from the basis Beto the basis Bsis
performed as a multiplication by a change of basis matrix:

~ v
Bs=
1
BsBe
~ v
Be,vx
vy
=^{^e1^{^e2
^|^e1^|^e2v1
v2
:
Note the change of basis matrix is actually an identity transformation. The
vector~ vremains unchanged—it is simply expressed with respect to a new
coordinate system. The change of basis from the Bs-basis to the Be-basis
is accomplished using the inverse matrix: Be\left[ 1\right]Bs= \left(Bs\left[ 1\right]Be\right)1.
B. Matrix representations of linear transformations
Bases play an important role in the representation of linear transforma-
tionsT:Rn!Rm. To fully describe the matrix that corresponds to some
linear transformation T, it is sufficient to know the effects of Tto then
vectors of the standard basis for the input space. For a linear transformation
T:R2!R2, the matrix representation corresponds to
MT=2
4j j
T\left(^{\right)T\left(^|\right)
j j3
52R22:
As a first example, consider the transformation xwhich projects vectors
onto thex-axis. For any vector ~ v= \left(vx;vy\right), we have x\left(~ v\right) = \left(vx;0\right).
The matrix representation of xis
Mx=
x1
0
x0
1
=1 0
0 0
:
As a second example, let’s find the matrix representation of R, the
counterclockwise rotation by the angle :
MR=
R1
0
R0
1
=cossin
sin cos
:
The first column of MRshows thatRmaps the vector ^{1\0to the
vector 1\= \left(cos;sin\right)T. The second column shows that Rmaps the
vector ^|= 1\
2to the vector 1\\left(
2\+\right) = \left(sin;cos\right)T.4
C. Dimension and bases for vector spaces
The dimension of a vector space is defined as the number of vectors
in a basis for that vector space. Consider the following vector space
S=spanf\left(1;0;0\right);\left(0;1;0\right);\left(1;1;0\right)g. Seeing that the space is described
by three vectors, we might think that Sis3-dimensional. This is not the
case, however, since the three vectors are not linearly independent so they
don’t form a basis for S. Two vectors are sufficient to describe any vector
inS; we can writeS=spanf\left(1;0;0\right);\left(0;1;0\right)g, and we see these two
vectors are linearly independent so they form a basis and dim\left(S\right) = 2 .
There is a general procedure for finding a basis for a vector space.
Suppose you are given a description of a vector space in terms of mvectors
V=spanf~ v1;~ v2;:::;~ vmgand you are asked to find a basis for Vand the
dimension ofV. To find a basis for V, you must find a set of linearly
independent vectors that span V. We can use the Gauss–Jordan elimination
procedure to accomplish this task. Write the vectors ~ vias the rows of a
matrixM. The vector space Vcorresponds to the row space of the matrix
M. Next, use row operations to find the reduced row echelon form \left(RREF\right)
of the matrix M. Since row operations do not change the row space of the
matrix, the row space of reduced row echelon form of the matrix Mis the
same as the row space of the original set of vectors. The nonzero rows in
the RREF of the matrix form a basis for vector space Vand the numbers
of nonzero rows is the dimension of V.
D. Row space, columns space, and rank of a matrix
Recall the fundamental vector spaces for matrices that we defined in
Section II-E: the column space C\left(A\right), the null spaceN\left(A\right), and the row
spaceR\left(A\right). A standard linear algebra exam question is to give you a
certain matrix Aand ask you to find the dimension and a basis for each
of its fundamental spaces.
In the previous section we described a procedure based on Gauss–Jordan
elimination which can be used “distill” a set of linearly independent vectors
which form a basis for the row space R\left(A\right). We will now illustrate this
procedure with an example, and also show how to use the RREF of the
matrixAto find bases forC\left(A\right)andN\left(A\right).
Consider the following matrix and its reduced row echelon form:
A=2
41 3 3 3
2 6 7 6
3 9 9 103
5 rref\left(A\right) =2
413 0 0
0 0 10
0 0 0 13
5:
The reduced row echelon form of the matrix Acontains three pivots. The
locations of the pivots will play an important role in the following steps.
The vectorsf\left(1;3;0;0\right);\left(0;0;1;0\right);\left(0;0;0;1\right)gform a basis forR\left(A\right).
To find a basis for the column space C\left(A\right)of the matrix Awe need
to find which of the columns of Aare linearly independent. We can
do this by identifying the columns which contain the leading ones in
rref\left(A\right). The corresponding columns in the original matrix form a basis
for the column space of A. Looking at rref \left(A\right)we see the first, third,
and fourth columns of the matrix are linearly independent so the vectors
f\left(1;2;3\right)T;\left(3;7;9\right)T;\left(3;6;10\right)Tgform a basis forC\left(A\right).
Now let’s find a basis for the null space, N\left(A\right)f~ x2R4jA~ x=~0g.
The second column does not contain a pivot, therefore it corresponds to a
free variable , which we will denote s. We are looking for a vector with three
unknowns and one free variable \left(x1;s;x 3;x4\right)Tthat obeys the conditions:
2
41 3 0 0
0 0 1 0
0 0 0 13
52
664x1
s
x3
x43
775=2
40
0
03
5\right)1x1\+ 3s= 0
1x3= 0
1x4= 0
Let’s express the unknowns x1,x3, andx4in terms of the free variable s.
We immediately see that x3= 0 andx4= 0, and we can write x1=3s.
Therefore, any vector of the form \left(3s;s;0;0\right), for anys2R, is in the
null space of A. We writeN\left(A\right) =spanf\left(3;1;0;0\right)Tg.
Observe that the dim\left(C\left(A\right)\right) = dim\left(R\left(A\right)\right) = 3 , this is known as the
rank of the matrix A. Also, dim\left(R\left(A\right)\right) \+ dim\left(N\left(A\right)\right) = 3 \+ 1 = 4 ,
which is the dimension of the input space of the linear transformation TA.E. Invertible matrix theorem
There is an important distinction between matrices that are invertible and
those that are not as formalized by the following theorem.
Theorem. For annnmatrixA, the following statements are equivalent:
1\right)Ais invertible
2\right)The RREF of Ais thennidentity matrix
3\right)The rank of the matrix is n
4\right)The row space of AisRn
5\right)The column space of AisRn
6\right)Adoesn’t have a null space \left(only the zero vector N\left(A\right) =f~0g\right)
7\right)The \determinant of Ais nonzero \det\left(A\right)6= 0
For a given matrix A, the above statements are either all true or all false.
An invertible matrix Acorresponds to a linear transformation TAwhich
maps then-dimensional input vector space Rnto then-dimensional output
vector space Rnsuch that there exists an inverse transformation T1
Athat
can faithfully undo the effects of TA.
On the other hand, an nnmatrixBthat is not invertible maps the
input vector space Rnto a subspaceC\left(B\right)\left( Rnand has a nonempty null
space. Once TBsends a vector ~ w2N\left(B\right)to the zero vector, there is no
T1
Bthat can undo this operation.
F . Determinants
The \determinant of a matrix, denoted \det\left(A\right)orjAj, is a special way to
combine the entries of a matrix that serves to check if a matrix is invertible
or not. The \determinant formulas for 22and33matrices area11a12
a21a22=a11a22a12a21; and
a11a12a13
a21a22a23
a31a32a33=a11a22a23
a32a33a12a21a23
a31a33\+a13a21a22
a31a32:
If thejAj= 0 thenAis not invertible. If jAj6= 0 thenAis invertible.
G. Eigenvalues and eigenvectors
The set of eigenvectors of a matrix is a special set of input vectors for
which the action of the matrix is described as a simple scaling . When
a matrix is multiplied by one of its eigenvectors the output is the same
eigenvector multiplied by a constant A~ e=~ e. The constant is called
aneigenvalue ofA.
To find the eigenvalues of a matrix we start from the eigenvalue equation
A~ e=~ e, insert the identity 1, and rewrite it as a null-space problem:
A~ e= 1~ e\right) \left(A 1\right)~ e=~0:
This equation will have a solution whenever jA 1j= 0. The eigenvalues
ofA2Rnn, denotedf1;2;:::;ng, are the roots of the characteristic
polynomial p\left(\right) =jA 1j. The eigenvectors associated with the
eigenvalueiare the vectors in the null space of the matrix \left(Ai 1\right).
Certain matrices can be written entirely in terms of their eigenvectors
and their eigenvalues. Consider the matrix that has the eigenvalues of
the matrixAon the diagonal, and the matrix Qconstructed from the
eigenvectors of Aas columns:
 =2
666666641 0
......0
0 0n3
77777775;Q=2
66666664j
~ e1~ enj3
77777775;thenA=QQ1:
Matrices that can be written this way are called diagonalizable.
The decomposition of a matrix into its eigenvalues and eigenvectors
gives valuable insights into the properties of the matrix. Google’s original
PageRank algorithm for ranking webpages by “importance” can be
formalized as an eigenvector calculation on the matrix of web hyperlinks.
VI. T EXTBOOK PLUG
If you’re interested in learning more about linear algebra, check out the
NO BULLSHIT GUIDE TO LINEAR ALGEBRA . The book is available via
lulu.com ,amazon.com , and also here: gum.co/noBSLA .